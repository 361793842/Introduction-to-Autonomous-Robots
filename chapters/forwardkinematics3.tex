
\section*{Take-home lessons}

\begin{itemize}
\item Forward kinematics are equivalent to finding a coordinate transform from a world coordinate system into a coordinate system on the robot. Such a transform is a combination of a (3x1) translation vector and a (3x3) rotation matrix that consists of the unit vectors of the robot coordinate system. Both translation and rotation can be combined into a 4x4 homogeneous transform matrix.
\item Forward and Inverse Kinematics of a mobile robot are performed with respect to the speed of the robot and not its position.
\item For calculating the effect of each wheel on the speed of the robot, you need to consider the contribution of each wheel independently.
\item Calculating the inverse kinematics analytically becomes quickly infeasible. You can then plan in configuration space of the robot using path-planning techniques.
\item The inverse kinematics of a robot involves solving the equations for the forward kinematics for the joint angles. This process is often cumbersome if not impossible for complicated mechanisms.
\item A simple numerical solution is provided by taking all partial derivatives of the forward kinematics in order to get an easily invertible expression that relates joint speeds to end-effector speeds.
The inverse kinematics problem can then be formulated as feedback control problem, which will move the end-effector towards its desired pose using small steps. Problems with this approach are local minima and singularities of the mechanism, which might render this solution infeasible.
\end{itemize}

\section*{Exercises}\small
\subsection*{Coordinate systems}
\begin{enumerate}
\item
\begin{enumerate}
 \item Write out the entries of a rotation matrix $^A_BR$ assuming basis vectors $X_A$, $Y_A$, $Z_A$, and $X_B$, $Y_B$, $Z_B$.
 \item Write out the entries of rotation matrix $^B_AR$.
 \end{enumerate}
\item Assume two coordinate systems that are co-located in the same origin, but rotated around the z-axis by the angle $\alpha$. Derive the rotation matrix from one coordinate system into the other and verify that each entry of this matrix is indeed the scalar product of each basis vector of one coordinate system with every other basis vector in the second coordinate system.
\item Consider two coordinate systems $\{B\}$ and $\{C\}$, whose orientation is given by the rotation matrix $^C_BR$ and have distance $^BP$. Provide the homogenous transform $^C_BT$ and its inverse $^B_CT$.
\item Consider the frame $\{B\}$ that is defined with respect to frame $\{A\}$ as $\{B\}=\{^A_BR, ^AP\}$. Provide a homogeneous transfrom from $\{A\}$ to $\{B\}$.
\end{enumerate}

\subsection*{Forward and inverse kinematics}
\begin{enumerate}
\item Consider a differential wheel robot with a broken motor, i.e., one of the wheels cannot be actuated anymore. Derive the forward kinematics of this platform. Assume the right motor is broken.
\item Consider a tri-cycle with two independent standard wheels in the rear and the steerable, driven front-wheel. Choose a suitable coordinate system and use $\phi$ as the steering wheel angle and wheel-speed $\dot{\omega}$. Provide forward and inverse kinematics.
\end{enumerate}
\normalsize
